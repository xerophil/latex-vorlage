\documentclass[
	paper=a4,
	fontsize=12pt,
	bibliography=totoc,
	listof=totoc,
	%draft,
	twoside,
]{scrreprt}

%%%%%%%%%%%%%%%%%%%%%%%%%%%%%%%%%%%%%%%%%%%%%%
%		Basis Präambel
%%%%%%%%%%%%%%%%%%%%%%%%%%%%%%%%%%%%%%%%%%%%%%

\usepackage[ngerman]{babel}
\usepackage[utf8]{inputenc}
\usepackage[T1]{fontenc}
	
%allows \enquote{} to easily generate locale quotation marks
\usepackage[babel,german=quotes]{csquotes} 


%%%%%%%%%%%%%%%%%%%%%%%%%%%%%%%%%%%%%%%%%%%%%%
%	Einstellungen der Titelsetie
%%%%%%%%%%%%%%%%%%%%%%%%%%%%%%%%%%%%%%%%%%%%%%

\title{\thetitle}
\subtitle{\thetopic}
\author{\theauthor}
\publishers{\thepublisher}



%%%%%%%%%%%%%%%%%%%%%%%%%%%%%%%%%%%%%%%%%%%%%%
%		Seitenabmessungen
%%%%%%%%%%%%%%%%%%%%%%%%%%%%%%%%%%%%%%%%%%%%%%

\usepackage[
	left=2.5cm,
	right=2.5cm,
	top=2.5cm,
	bottom=2.5cm,
	bindingoffset=1cm,
	includehead,
	includefoot
]{geometry}

%%%%%%%%%%%%%%%%%%%%%%%%%%%%%%%%%%%%%%%%%%%%%%
%		Parameterlose Packete
%%%%%%%%%%%%%%%%%%%%%%%%%%%%%%%%%%%%%%%%%%%%%%

\usepackage[table]{xcolor}
\usepackage{
	scrhack,		%supresses warning \float@addtolists detected!
%	fancyhdr,
	graphicx,
%	amssymb,
%	makeidx,
	listings,
	lmodern,
	wrapfig,		%wrapping stuff
	subfig,			%allows you to put multiple pics into one figure
	blindtext,		%generates some random placeholder text
	lastpage,		
	paralist,		%provides compact itemize 
	color,
	MnSymbol,
	pdfpages,
	todonotes,
	epigraph,		%inspirational quote at the beginning of a chapter
	refcheck,		%finds unused references. useful for bibliography 
	booktabs,
	multirow,
}

\shorthandoff{"}


%%%%%%%%%%%%%%%%%%%%%%%%%%%%%%%%%%%%%%%%%%%%%%
%		Farben
%%%%%%%%%%%%%%%%%%%%%%%%%%%%%%%%%%%%%%%%%%%%%%

\definecolor{dkgreen}{rgb}{0,0.6,0}
\definecolor{gray}{rgb}{0.5,0.5,0.5}
\definecolor{mauve}{rgb}{0.58,0,0.82}
\definecolor{lightgray}{gray}{0.75}
 
 
%%%%%%%%%%%%%%%%%%%%%%%%%%%%%%%%%%%%%%%%%%%%%%
%		Einstellungen für Listings
%%%%%%%%%%%%%%%%%%%%%%%%%%%%%%%%%%%%%%%%%%%%%%


 
\lstset{ %
	%extendedchars=\true,
	%inputencoding=utf8,
	floatplacement=htb,
	language=Java,
	basicstyle=\footnotesize\ttfamily,	%quelltext ist etwas kleiner als normaler text
	numbers=left,
	numberstyle=\tiny\color{gray},		%kleine, graue Zeilennummern
	stepnumber=1,
	numbersep=5pt,                  	%Entfernung der Zeilennummern zum Code
	backgroundcolor=\color{white},
	showspaces=false,               	%keine Unterstriche für Leerzeichen
	showstringspaces=false,         	%auch nicht für Strings
	showtabs=false,                 	%und nicht für tabs
	frame=single,
	rulecolor=\color{black}, 			%Achtung muss gesetzt sein, weils sonst bunte
										%rahmen gibt.
	tabsize=2,
	captionpos=b,						%Beschriftung unter dem Code
	breaklines=true,					%zeilen Umbrechen, ragen sonst über Rand hinaus
	breakatwhitespace=false, 			%hartes Umbrechen, folgende Zeilen fügen aber
										%entsprechende Markierungen hinzu
	prebreak=\raisebox{0ex}[0ex][0ex]{\color{black}\ensuremath{\rhookswarrow}},
	postbreak=\raisebox{0ex}[0ex][0ex]{\color{black}\ensuremath{\rcurvearrowse}},
	title=\lstname,						%Zeigt Dateinamen wenn keine caption angegeben
	keywordstyle=\color{blue},			%blaue Keywörter, wie void oder null
	commentstyle=\color{dkgreen},		%grüne Kommentare
	stringstyle=\color{mauve}, 			%und die Strings 
	escapeinside={\%*}{*)},				%ermöglicht kommentare im Listing mit %
	morekeywords={create,Void},			%zusätzliche keywords
}

%sorgt dafür, dass man in listings auch umlaute benutzen kann. dies
%geht leider nicht wenn die dokumente utf8 kodiert sind. verwendet man
%latin1, gibt es keine probleme.
\lstset{literate=%
	{Ö}{{\"O}}1
	{Ä}{{\"A}}1
	{Ü}{{\"U}}1
	{ß}{{\ss}}2
	{ü}{{\"u}}1
	{ä}{{\"a}}1
	{ö}{{\"o}}1
}

%irgend ein komisches Makro, was dafür sorgt, dass über lstinline eingefügter
%code die selbe textgröße hat wie der normale text, weil
%normale listings etwas kleiner in der schriftgröße sind
%http://www.latex-community.org/forum/viewtopic.php?f=5&t=2072
\makeatletter
	\lst@AddToHook{TextStyle}{\let\lst@basicstyle\normalsize\ttfamily}
\makeatother


%%%%%%%%%%%%%%%%%%%%%%%%%%%%%%%%%%%%%%%%%%%%%%
%		Einstellungen für PDF
%%%%%%%%%%%%%%%%%%%%%%%%%%%%%%%%%%%%%%%%%%%%%%

\usepackage[colorlinks=true,
	breaklinks=false,
	linktocpage=false,
	linkcolor=blue,
	citecolor=blue,
	filecolor=black,
	urlcolor=blue,
	frenchlinks=false,
	bookmarks=true,
	bookmarksopen=true,
	bookmarksopenlevel=3,
	plainpages=false,
	pdfpagelabels=true,
	pdftitle={\thetitle},
	pdfauthor=\theauthor,
	pdfsubject={\thetopic},
]{hyperref}

\pdfinfo{
    /Author (\theauthor)
    /Title (\thetitle)
    /Subject (\thetopic)
}


%%%%%%%%%%%%%%%%%%%%%%%%%%%%%%%%%%%%%%%%%%%%%%
%		Header/Footer der Seiten
%%%%%%%%%%%%%%%%%%%%%%%%%%%%%%%%%%%%%%%%%%%%%%

\usepackage[
	headsepline,
	footsepline,
	automark,
]{scrpage2}

% Festlegung Kopf- und Fußzeile     
\defpagestyle{meinstil}{%
	{\headmark \hfill}
	{\hfill \headmark}
	{\hfill \headmark\hfill }
	(\textwidth,.4pt)
}{%
	(\textwidth,.4pt)
	{\pagemark\hfill\thefaculty}
	{\theauthor\hfill\pagemark}
	{\hfill\pagemark} 
}
\pagestyle{meinstil} 



%%%%%%%%%%%%%%%%%%%%%%%%%%%%%%%%%%%%%%%%%%%%%%
%		Eigene Befehle
%%%%%%%%%%%%%%%%%%%%%%%%%%%%%%%%%%%%%%%%%%%%%%

%aquote für zitat mit quelle rechts unten
%siehe http://tex.stackexchange.com/questions/13756/quote-environment-with-reference-at-the-end-right

\def\signed #1{{\leavevmode\unskip\nobreak\hfil\penalty50\hskip2em
  \hbox{}\nobreak\hfil(#1)%
  \parfillskip=0pt \finalhyphendemerits=0 \endgraf}}

\newsavebox\mybox
\newenvironment{aquote}[1]
  {\savebox\mybox{#1}\begin{quotation}}
  {\signed{\usebox\mybox}\end{quotation}}

