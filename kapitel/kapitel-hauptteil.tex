\chapter{Hauptteil}
\section{Figures und Floats}
\begin{wrapfigure}{L}{0.46\textwidth}
  \centering
	\includegraphics[]{figures/SampleAppProjektbaum}
    \caption{Projektbaum eines GWT Moduls}    
  \label{fig:projektbaum}
\end{wrapfigure}

\blindtext[2]

\begin{figure}[ht]
  \centering
	\includegraphics[width=\textwidth]{figures/DomainModel}
    \caption[Domänenmodell der Aufgabenpoolverwaltung]{Domänenmodell der Aufgabenpoolverwaltung.\\ \textbf{Quelle:} Prof. Dr. Robert Garmann}
  \label{fig:domainmodel}
\end{figure}

\begin{wrapfigure}{r}{0.45\textwidth}
  \vspace{-25pt}
  \centering
  \subfloat[Klassiches MVP]{\label{fig:mvp1}\includegraphics[scale=0.5]{figures/mvp1}}
  \quad
  \subfloat[MVP in der GWT-Architektur]{\label{fig:mvp2}\includegraphics[scale=0.5]{figures/mvp2}}
    \caption{MVP nach \cite{GuUn2010}}
  \label{fig:mvp}  
  \vspace{-15pt}
\end{wrapfigure}

\blindtext[2]

\section{Listings}

Code wird für Java hervorgehoben. Ebenso inline-code wie dieser: \lstinline|public void essen() { System.out.println("esse"); }|

\lstinputlisting[float,label=lst:requestfactory,caption={Verwendung der RequestFactory beim Client\cite{GWT_REQ}}]{code/requestfactory.java}


\lstinputlisting[label=lst:uibinder_xml,language=xml,caption={Die XML-Beschreibung eines einfachen Widgets}]{code/HelloWidgetWorld.ui.xml}

Die Angabe von \lstinline|float| in den Attributen von \lstinline|\lstinputlisting| sorgt dafür, dass Quelltext nicht umgebrochen wird. 


\section{Sonstiges}
\subsection{Listen}
\begin{itemize}
\item eins
\item zwei
\item drei
\end{itemize}

\begin{enumerate}
\item eins
\item zwei
\item drei
\end{enumerate}

\subsection{Definitionen}
\begin{description}
\item[eins] ne zahl
\item[zwei] noch ne zahl
\item[drei] und noch eine
\end{description}

\subsection{Kompakte listen}
Die normalen Auflistungen haben viel whitespace oben und unten und zwischen drin. Mit \texttt{compactitem} kann dieser platz....
\begin{compactitem}
\item eins
\item zwei
\item drei
\end{compactitem}
.... verringert werden

\subsection{Todo Notes}
\todo{Referenz finden} Todonotes sind nützlich um Notizen im Fließtext zu platzieren. Können auch über eine ganze Zeile gehen
\todo[inline]{Test}

\section{Tabellen}
\blindtext
\begin{table}[h]
	\centering
	\begin{tabular}{|l|l|l|}
\hline 
		Metrik 		  &Beschreibung 								&  Brauchbarkeit\\ 
\hline  
		\textbf{WMC}  &Gewichtete Methoden pro Klasse 				&  mäßig 		\\ 
\hline  
		\textbf{DIT}  &Tiefe im Vererbungsbaum 						&  hoch 		\\ 
\hline  
		\textbf{RFC}  &Antwortmenge einer Klasse 					&  hoch 		\\ 
\hline 
		\textbf{NOC}  &Zahl von Unterklassen 						&  hoch 		\\ 
\hline  
		\textbf{LCOM} &Mangel an Zusammenhang zwischen Methoden 	&  niedrig  	\\ 
\hline  
		\textbf{CBO}  &Kopplung zwischen Objektklassen 				&  hoch  		\\ 
\hline 
	\end{tabular} 
	\caption{Kennzahlen nach Chidamber \& Kemerer (aus \cite{Prech1999})}
	\label{tab:metrik}
\end{table}


\section{Zitate}

Inline-Zitate stehen \enquote{\emph{mitten im Text}} und sind üblicherweise kursiv gedruckt, außerdem sind sie in den Satz eingebunden.

Abgesetzte Zitate werden entweder mit der quote-Umgebung für kurze Zitate gemacht, oder mit der qoutation-Umgebung für längere Zitate

\begin{quote}
Nachts ist es kälter als Draußen
\end{quote}


\begin{quotation}
Hiermit erkläre ich an Eides Statt, dass ich die eingereichte Masterarbeit
				selbständig und ohne fremde Hilfe verfasst, andere als die von mir angegebenen Quellen
				und Hilfsmittel nicht benutzt und die den benutzten Werken wörtlich oder
				inhaltlich entnommenen Stellen als solche kenntlich gemacht habe. 
\end{quotation}

Da diese häufig eine Quellenangabe benötigen und diese häufig unten rechts vom Zitat zu finden ist, habe ich eine aquote-Umgebung aus dem Interwebz eingefügt:

\begin{aquote}{Eidesstattliche Erklärung der Hochschule}
Hiermit erkläre ich an Eides Statt, dass ich die eingereichte Masterarbeit
				selbständig und ohne fremde Hilfe verfasst, andere als die von mir angegebenen Quellen
				und Hilfsmittel nicht benutzt und die den benutzten Werken wörtlich oder
				inhaltlich entnommenen Stellen als solche kenntlich gemacht habe. 
\end{aquote}

\begin{aquote}{Unbekannt}
Nachts ist es kälter als Draußen
\end{aquote}
